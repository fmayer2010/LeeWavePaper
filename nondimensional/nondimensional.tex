\documentclass[12pt]{article}

\usepackage{natbib,amsfonts,graphics,amsmath}

%
% New command file
%

% Partial Derivative
\newcommand{\D}[2]{\frac{\partial{#1}}{\partial{#2}}}
\newcommand{\dd}[2]{\frac{d{#1}}{d{#2}}}
\newcommand{\ND}[3]{\frac{\partial^{#3}{#1}}{\partial{#2}^{#3}}}
\newcommand{\DHO}[2]{\frac{\partial^H{#1}}{\partial{#2}}}
\newcommand{\de}[2]{\frac{\delta{#1}}{\delta{#2}}}
\newcommand{\deln}[3]{\frac{\delta^#3 #1}{\delta #2^#3}}
\newcommand{\SD}[2]{\frac{\partial^2{#1}}{\partial{#2}\partial{#2}}}
\newcommand{\order}[1]{\mathcal{O}\left(#1\right)}
\newcommand{\etal}{{\it\space et al. }}
\newcommand{\ub}{\mathbf{u}}
\newcommand{\fb}{\mathbf{f}}
\newcommand{\rhobar}{\overline{\rho}}
\newcommand{\xb}{\mathbf{x}}
\newcommand{\ut}{\utilde{u}}
\newcommand{\ol}[1]{\overline{#1}}
\newcommand{\twocol}[2]{\parbox{2in}{#1}\parbox{2in}{#2}}
%
% \insertfig{scalefactor}{epsfile}{caption}{label}
%
\newcommand{\insertfig}[4]{
	\begin{figure}[ht!]
	\centerline{
	\scalebox{#1}{\includegraphics{#2}}
	}
	\caption{#3}
	\label{#4}
	\end{figure}}


\begin{document}

\title{Derivation of the nondimensional lee wave equations}

\author{Oliver Fringer}

\maketitle

\section{Dimensional analysis}

The form drag $F$ (per unit width) due to a hill of height $h_0$ in an infinitely 
deep fluid is characterized by the dimensional quantities
$U$, $k$, $N$, and $h_0$.  Choosing $U$ and $N$ to nondimensionalize $k$ and $h_0$, the governing
nondimensional parameters are $J = N h_0/U$ and $\epsilon = k U/N$.  Since the units of the form drag
are length$^3$~time$^{-2}$, then a scale for $F$ in terms of $U$ and $N$ is $\rho_0 U^3 N^{-1}$. Therefore,
the form drag must satisfy
\[
\frac{F}{\rho_0 U^3 N^{-1}} = f(J, \epsilon)\,.
\]

\section{Nondimensional equations} \label{eq:equations}

Let $\ub_{\mbox{total}} = U {\bf e}_x + \ub$, $\rho_{\mbox{total}} = \rhobar(z) + \rho$, and
$p_{\mbox{total}} = \rho_0 \overline{p}(z) + \rho_0 p$ (the total notation is
to avoid carrying around the prime, but in a paper you should carry around the prime).
The steady momentum and density transport equations are given by
\begin{eqnarray}
U \D{u}{x} + \ub\cdot\nabla u &=& -\D{p}{x}\,,\\
U \D{w}{x} + \ub\cdot\nabla w &=& -\D{p}{z} - \frac{\rho}{\rho_0} g\,,\\
U \D{\rho}{x} + \ub\cdot\nabla\rho &=& \frac{\rho_0 N^2}{g} w\,,
\end{eqnarray}
where $N^2 = -g/\rho_0 \partial\rhobar/\partial z$, subject to continuity $\nabla\cdot\ub=0$ and
the kinematic bottom boundary condition
\[
U\D{h}{x} + u \D{h}{x} = w\,.
\]  
Nondimensionalize with
\begin{eqnarray*}
u &=& u_0 u^*\,,\\
w &=& w_0 w^*\,,\\
\rho &=& R \rho^*\,,\\
p &=& P p^*\,,\\
x &=& k^{-1} x^*\,,\\
z &=& \delta z^*\,.
\end{eqnarray*}
Nondimensionalizing the kinematic bottom boundary condition gives
(after ignoring the $*$)
\[
k U h_0 \D{h}{x} + k u_0 h_0 \D{h}{x} = w_0 w\,.
\]
If we require a balance between the linear terms, this implies $w_0 = k h_0 U$, so that
\[
\D{h}{x} + F \D{h}{x} = w\,,
\]
where $F = u_0/U$ is a Froude number. Now, the vertical scale of the flow as given by $\delta$ is not the
same as the hill height $h_0$, since $\delta$ must be finite as $h_0\to 0$ (the linear limit). The 
vertical scale is thus dictated by continuity, which requires
\[
k u_0 \D{u}{x} + \frac{w_0}{\delta} \D{w}{z} = 0\,,
\]
or, since this implies $k u_0 = w_0/\delta$, then we must have $\delta = w_0/(k u_0) = k h_0 U/(k u_0) = F^{-1} h_0$.
Nondimensionalizing the $u$-momentum equation,
\begin{eqnarray}
k u_0 U \D{u}{x} + k u_0^2\ub\cdot\nabla u  &=& -k P \D{p}{x}\,.
\end{eqnarray}
Since we require a leading-order balance between the pressure gradient and the linear momentum advection term,
we must have $P = u_0 U$, which gives
\begin{eqnarray}
\D{u}{x} + F \ub\cdot\nabla u &=& -\D{p}{x}\,.
\end{eqnarray}
The nondimensional density transport equation is given by
\[
k U R \D{\rho}{x} + k u_0 R \ub\cdot\nabla\rho = \frac{k \rho_0 h_0 N^2 U}{g} w\,.
\]
If we require a balance between the linear advection terms, then the scale for the density
perturbation is 
\[
R = \frac{\rho_0 N^2 h_0}{g}\,,
\]
so that the nondimensional density equation is
\[
\D{\rho}{x} + F \ub\cdot\nabla\rho = w\,.
\]
Nondimensionalizing the vertical momentum equation, we have
\[
k^2 h_0 U^2 \D{w}{x} + k^2 h_0 u_0^2\ub\cdot\nabla w = -\frac{P}{\delta}\D{p}{z} - \frac{g R}{\rho_0} \rho\,.
\]
If we require a vertical hydrostatic balance to leading order, then we must have 
\[
\frac{P}{\delta} = \frac{g R \delta}{\rho_0} 
  = N^2 h_0\,,
\]
and
\[
P = \frac{g R \delta}{\rho_0} 
  = \frac{g}{\rho_0} \frac{\rho_0 N^2 h_0}{g} \frac{h_0}{F}
  = \frac{g}{\rho_0} \frac{\rho_0 N^2 h_0}{g} \frac{U}{N}
  = U N h_0\,.
\]
which gives
\[
\epsilon^2 \left(\D{w}{x} + F\ub\cdot\nabla w\right) = -\D{p}{z} - \rho\,,
\]
where 
\[
\epsilon = \frac{k U}{N}
\]
is the nonhydrostatic parameter. Now, returning to the pressure, since from the vertical momentum equation we
require $P = U N h_0$ and from the horizontal momentum equation we require $P = u_0 U$, then equating the
two implies that $u_0 = N h_0$ and the Froude number is given by
\[
F = \frac{u_0}{U} = \frac{N h_0}{U} = J\,,
\]
where
\[
J = \frac{N h_0}{U}\,.
\]
Therefore, in terms of $J$, the governing nondimensional equations are given by
\begin{eqnarray*}
\D{u}{x} + J\ub\cdot\nabla u &=& -\D{p}{x}\,,\\
\epsilon^2 \left(\D{w}{x} + J\ub\cdot\nabla w\right) &=& -\D{p}{z} - \rho\,,\\
\D{\rho}{x} + J \ub\cdot\nabla\rho &=& w\,,
\end{eqnarray*}
subject to $\nabla\cdot\ub=0$ and the kinematic bottom boundary condition
\[
\left(1 + J u\right)\D{h}{x} = w\,.
\]
These nondimensional equations are consistent with the original nondimensionalization which implied
that the problem is uniquely characterized by $\epsilon$ and $J$.
The relevant scales (nondimensionalized by $N$ and $U$) are given by
\begin{eqnarray*}
\frac{u_0}{U} &=& J\,,\\
\frac{w_0}{U} &=& \epsilon J\,,\\
\frac{gR}{\rho_0 U N} &=& J\,,\\
\frac{P}{U^2} &=& J\,,\\
\frac{\delta N}{U} &=& 1\,.
\end{eqnarray*}

Contrary to what is stated in the literature, we can show that it is
in fact appropriate to refer to $J$ as an internal Froude number. If we define
$Fr_\delta = u_0/\sqrt{g' \delta}$, where $g'\delta =g R/\rho_0 \delta = J U^2$, this gives
$Fr_{\delta} = J^{1/2}$. Therefore, $J^{1/2}$ can be thought of as 
the ratio of the inertial to gravitational forces arising from the perturbed
flow above the hill. Although we would expect a larger $N$ to block the flow and
reduce the magnitude of the perturbation above the hill, the scaling shows that
$u_0 = J U$, implying that the perturbation velocity above the sill increases with
increasing $J$. However, the gravitational force resulting from the perturbation is
given by $\sqrt{g'\delta} = J^{1/2} U$, which grows more slowly than $u_0$ with
increasing $J$.

\section{Linear, nonhydrostatic equations}

The governing equations in the linear limit $J\to 0$ are given by
\begin{eqnarray*}
\D{u}{x} &=& -\D{p}{x}\,,\\
\epsilon^2 \D{w}{x} &=& -\D{p}{z} - \rho\,,\\
\D{\rho}{x} &=& w\,.
\end{eqnarray*}
The governing equation for $w$ is then given by
\[ 
\ND{w}{z}{2} + \epsilon^2 \ND{w}{x}{2} + w = 0\,.
\]
Assume a sinusoidal topography such that $h(x) = \sin(x)$.

\subsection{Propagating solution}

When $\epsilon<1$, the hydrostatic solution is of the form
$w(x,z) = \cos(x + m z)$, which implies $m = (1-\epsilon^2)^{1/2}$.  Substitution into
the governing linear equations gives
\begin{eqnarray*}
u(x,z) &=& -m \cos(x + m z)\,,\\
w(x,z) &=& \cos(x + m z)\,,\\
\rho(x,z) &=& \sin(x + m z)\,,\\
p(x,z) &=& m \cos(x + m z)\,.
\end{eqnarray*}
The dimensional form drag (per unit width) over one wavelength is given by (here, $*$ implies dimensional
quantities)
\[
F^* = \int_{0}^{2\pi/k^*} p^*(x^*,z^*=0)\D{h^*}{x^*}\, dx^*\,.
\]
Nondimensionalizing gives
\[
\frac{F^*}{\rho_0 U^3 N^{-1}} = J^2 \int_{0}^{2\pi} p(x,z=0)\D{h}{x}\, dx\,.
\]
Substitution of $p$ and $h$ then gives
\[
\frac{F^*}{\rho_0 U^3 N^{-1}} = \pi J^2\left(1 - \epsilon^2\right)^{1/2}\,.
\]


\subsection{Evanescent solution}

When $\epsilon>1$, the nonhydrostatic solution is of the form
$w(x,z) = \cos(x)\exp(-m z)$, which implies $m = (\epsilon^2-1)^{1/2}$.  Substitution into
the governing linear equations gives
\begin{eqnarray*}
u(x,z) &=& m \sin(x) \exp(-m z)\,,\\
w(x,z) &=& \cos(x) \exp(-m z)\,,\\
\rho(x,z) &=& \sin(x) \exp(-m z)\,,\\
p(x,z) &=& -m \sin(x) \exp(-m z)\,.
\end{eqnarray*}
Since $p$ and $w$ are $\pi/2$ out of phase, the form drag is identically zero.
 
\end{document}

w = cos(x)exp(-m z)
wx = -sin(x)exp(-m z)
ux = -wz = m cos(x) exp(-m z)
u = m sin(x) exp(-m z)

rhox = w = cos(x)exp(-m z)
rho = sin(x)exp(-m z)

e2 wx = -pz - rho
pz = -e2 wx - rho = e2 sin(x) exp(-m z) - sin(x) exp(-m z) = m^2 sin(x) exp(-m z)
p = (-1/m) m^2 sin(x) exp(-m z) = -m sin(x) exp(-m z)



